% =====================================
% CONFIGURATION DES PACKAGES ET COULEURS
% =====================================

% --- Encodage et langue ---
\usepackage[utf8]{inputenc}
\usepackage[T1]{fontenc}
\usepackage[french]{babel}
\renewcommand{\familydefault}{\sfdefault}  % Utilise Computer Modern Sans Serif
\renewcommand{\sfdefault}{qag}  % Spécifie cmss comme police sans empattement

% --- Packages pour la mise en page et le contenu ---
\usepackage{graphicx}               % Pour inclure des images
\usepackage[a4paper, margin=2.5cm]{geometry} % Gestion des marges
\usepackage{amsmath, amssymb}       % Outils mathématiques avancés
\usepackage[
    hidelinks,
    pdftitle={TRENCHANT\_TROULLIER\_VIRQUIN\_GES7\_Projet},
    pdfauthor={TRENCHANT Evan, TROULLIER Laël, VIRQUIN Rudy},
    pdfsubject={Projet GE S7},
    pdfkeywords={génie électrique, projet S7, INSA Strasbourg, GES7},
    pdfcreator={LaTeX},
    pdfproducer={LaTeX}
]{hyperref}    % Pour les liens cliquables et métadonnées PDF
\usepackage{fancyhdr}               % Pour personnaliser les en-têtes et pieds de page
\usepackage{xcolor}                 % Pour utiliser des couleurs
\usepackage{tcolorbox}              % Pour créer des boîtes colorées
\usepackage{float}                  % Pour mieux placer les figures
\usepackage{caption}                % Pour personnaliser les légendes
\usepackage{subcaption}             % Pour des sous-figures
\usepackage{booktabs}               % Pour de jolis tableaux
\usepackage{titlesec}               % Pour personnaliser les titres de chapitres
\usepackage{lastpage}               % Pour obtenir le nombre total de pages
\usepackage{datetime}               % Pour le formatage des dates

% --- Package CircuiTikz avec configuration spéciale ---
\usepackage{tikz}                   % TikZ doit être chargé avant circuitikz
\usepackage[siunitx]{circuitikz} % Configuration pour les diagrammes électriques
\usetikzlibrary{babel}

% --- Définition des couleurs personnalisées ---
\definecolor{INSAbleu}{RGB}{0, 70, 127}
\definecolor{boxgray}{gray}{0.95}
\definecolor{remindercolor}{RGB}{173, 216, 230}     % Bleu clair pour les rappels
\definecolor{conclusioncolor}{RGB}{220, 53, 69}     % Rouge pour les conclusions
\definecolor{conclusionbg}{RGB}{255, 240, 240}     % Fond rouge très clair
\definecolor{BLANC}{RGB}{255, 255, 255}            % Couleur blanche pour le pied de page

% --- Définition des modèles de boîtes colorées ---
\newtcolorbox{reminderbox}[1][Rappel]{
    colback=remindercolor!20,
    colframe=remindercolor!80,
    coltitle=white,
    colbacktitle=remindercolor!80,
    title=#1,
    fonttitle=\bfseries,
    rounded corners,
    boxrule=1pt,
    left=3mm,
    right=3mm,
    top=2mm,
    bottom=2mm
}

\newtcolorbox{conclusionbox}[1][Conclusion]{
    colback=conclusionbg,
    colframe=conclusioncolor,
    coltitle=white,
    colbacktitle=conclusioncolor,
    title=#1,
    fonttitle=\bfseries,
    rounded corners,
    boxrule=1pt,
    left=3mm,
    right=3mm,
    top=2mm,
    bottom=2mm
}

\newtcolorbox{infobox}[1][Information]{
    colback=INSAbleu!10,
    colframe=INSAbleu,
    coltitle=white,
    colbacktitle=INSAbleu,
    title=#1,
    fonttitle=\bfseries,
    rounded corners,
    boxrule=1pt,
    left=3mm,
    right=3mm,
    top=2mm,
    bottom=2mm
}

\newtcolorbox{classicbox}[1][Information]{
    colback=gray!10,
    colframe=gray!50,
    title=#1,
    fonttitle=\bfseries,
    rounded corners,
    boxrule=1pt,
    left=3mm,
    right=3mm,
    top=2mm,
    bottom=2mm
}