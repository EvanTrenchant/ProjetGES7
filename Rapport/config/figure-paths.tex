% Configuration pour la gestion automatique des chemins de figures
% Auteur: Assistant LaTeX
% Date: 2026-01-15

\usepackage{etoolbox}
\usepackage{xstring}

\makeatletter % Activation des @ pour les commandes internes

% Liste globale pour stocker les paires {Label/Texte}{Chemin}
\newcommand{\figurepathlist}{}

% --- 1. Ajout à la liste ---
\newcommand{\addtofigpathlist}[2]{%
    \listgadd{\figurepathlist}{{#1}{#2}}%
}

% --- 2. Rétrocompatibilité \figpath ---
\newcommand{\figpath}[2]{%
    \addtofigpathlist{Figure \ref{#1}}{#2}%
    \expandafter\gdef\csname figpath@#1\endcsname{#2}%
}

% --- 3. Redéfinition de \figref ---
% Cette commande est utilisée dans le texte.
% Elle affiche le label, met à jour le footer, et écrit dans le .aux
\renewcommand{\figref}[2]{%
    % Ne rien afficher inline — enregistrer uniquement pour le footer et le .aux
    \ifdefempty{\figfooter}{%
        \gdef\figfooter{\tiny $\blacktriangleright$ \textbf{#1:} \texttt{#2}}%
    }{%
        \gappto\figfooter{\\[2pt]\tiny $\blacktriangleright$ \textbf{#1:} \texttt{#2}}%
    }%
    % Enregistrement dans le fichier .aux (pour apparition dans la liste finale)
    \protected@write\@auxout{}{%
        \string\registerfigpathfromaux{\unexpanded{#1}}{\unexpanded{#2}}%
    }%
}

% Commande exécutée lors de la lecture du .aux
\newcommand{\registerfigpathfromaux}[2]{%
    \addtofigpathlist{#1}{#2}%
}

% --- 4. Affichage de la liste finale ---
\newcommand{\processfigpathItem}[1]{\processfigpathItemSplit#1}

\newcommand{\processfigpathItemSplit}[2]{%
    % Protection supplémentaire pour l'affichage dans la liste
    % Si l'utilisateur a mis des '_' non échappés par erreur, on les corrige
    \noexpandarg%
    \StrSubstitute{#2}{_}{\_}[\safePathDisplay]%
    \item[] \textbf{#1:} \texttt{\safePathDisplay}%
}

\newcommand{\printfigurelist}{%
    \ifdefempty{\figurepathlist}{%
        \item[] \textit{Aucune figure avec chemin défini.}%
    }{%
        \forlistloop{\processfigpathItem}{\figurepathlist}%
    }%
}

% --- 5. Helper pour récupérer un chemin ---
\newcommand{\getfigpath}[1]{%
    \ifcsdef{figpath@#1}{%
        \csuse{figpath@#1}%
    }{%
        \textit{Chemin non défini}%
    }%
}

\makeatother
