% Configuration pour la gestion automatique des chemins de figures
% Auteur: Assistant LaTeX
% Date: 2025-11-08

% Packages requis
\usepackage{etoolbox}

% Création d'une séquence pour stocker les labels des figures
\newcommand{\figurepathlist}{}

% Commande pour associer un chemin à une figure
% Usage: \figpath{label}{chemin}
% Exemple: \figpath{fig:mon_schema}{Modelisation/01_12_09_2025/schema.slx}
\newcommand{\figpath}[2]{%
    \expandafter\gdef\csname figpath@#1\endcsname{#2}%
    \ifdefempty{\figurepathlist}{%
        \gdef\figurepathlist{#1}%
    }{%
        \xappto\figurepathlist{,#1}%
    }%
}

% Commande pour afficher le chemin d'une figure
% Usage: \getfigpath{label}
\newcommand{\getfigpath}[1]{%
    \ifcsdef{figpath@#1}{%
        \csuse{figpath@#1}%
    }{%
        \textit{Chemin non défini}%
    }%
}

% Commande auxiliaire pour traiter chaque élément de la liste
\newcommand{\processfigureitem}[1]{%
    \item[] \textbf{Figure \ref{#1}:} \texttt{\getfigpath{#1}}
    
}

% Commande pour imprimer toute la liste des figures avec leurs chemins
\newcommand{\printfigurelist}{%
    \ifdefempty{\figurepathlist}{%
        \item[] \textit{Aucune figure avec chemin défini.}%
    }{%
        \expandafter\forcsvlist\expandafter\processfigureitem\expandafter{\figurepathlist}%
    }%
}
