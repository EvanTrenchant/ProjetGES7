\chapter{Dimensionnement des asservissements}

Après avoir validé la structure de contrôle avec des régulateurs PI idéaux en simulation, il est nécessaire de passer à l'étape de réalisation pratique. Cette phase consiste à traduire les fonctions de transfert théoriques en circuits électroniques réels à base d'amplificateurs opérationnels (AOP). 

Le dimensionnement des asservissements constitue une étape cruciale du projet, car elle établit le lien entre la théorie et la mise en œuvre physique. Chaque bloc de la chaîne de commande doit être soigneusement dimensionné afin de garantir que les performances obtenues en simulation soient effectivement reproduites sur le système réel. Dans ce chapitre, nous détaillons le dimensionnement de chacun des éléments constitutifs de la boucle d'asservissement, en commençant par les circuits soustracteurs, puis en abordant les correcteurs PI pour les boucles de courant et de vitesse.

\section{Réalisation des soustracteurs}

\noindent
\begin{minipage}[t]{0.56\textwidth}
\vspace{0pt}
Pour commencer nous réalisons un montage soustracteur permettant de comparer la sortie du montage a la commande. 

On utilise alors ce montage a AOP qui permet, en choisissant les mêmes valeurs de résistances, d'obtenir un soustracteur simple:

\begin{equation*}
    V_{S} = V_2 - V_1
\end{equation*}

\vspace{0.5cm}
En  pratique, on choisit des résistances de $10k\Omega$.
\end{minipage}
\hfill
\begin{minipage}[t]{0.4\textwidth}
\vspace{-80pt}
\begin{figure}[H]
    \centering
    \includegraphics[width=\textwidth]{images/Choix_des_composants/Soustracteur.png}
    \caption{\scriptsize{Montage soustracteur à AOP}}
    \label{fig:soustracteur_aop}
\end{figure}
\end{minipage}


\section{Dimensionnement des correcteurs réels}

\subsection{Structure du correcteur PI}
La structure du correcteur PI à implémenter est représentée à la figure \ref{fig:schema_pi_aop}.

\begin{figure}[H]
    \centering
    \begin{circuitikz}[european resistors, scale=1.1, transform shape]
        % Premier étage - Correcteur H1(s)
        % Résistance d'entrée R connectée à l'entrée - de l'AOP1
        \draw (0,0) to[R, l=$R$, o-] (2,0);
        
        % Premier AOP (borne - en haut)
        \draw (2,0) node[op amp, anchor=-](OA1){};
        
        % Résistance et condensateur EN SÉRIE dans la rétroaction
        \draw (OA1.-) |- ++(0,1.2) coordinate(top1);
        \draw (top1) to[R, l=$R$] ++(1.5,0) coordinate(midRC);
        \draw (midRC) to[C, l=$C$] ++(1.5,0) coordinate(endRC);
        \draw (endRC) |- (OA1.out);
        
        % Masse sur l'entrée +
        \draw (OA1.+) -- ++(0,-0.4) node[ground]{};
        
        % Sortie du premier étage avec noeud Va aligné
        \draw (OA1.out) to[short, -*] ++(0.63,0) coordinate(va);
        \draw (va) node[below=2mm]{$V_a$};
        
        % Deuxième étage - Correcteur H2(s)
        % Résistance R1 reliée sur la borne - de l'AOP2
        \draw (va) to[R, l=$R1$] ++(2,0) coordinate(beforeAOP2);
        
        % Second AOP (borne - en haut)
        \draw (beforeAOP2) node[op amp, anchor=-](OA2){};
        
        % Résistance R2 dans la rétroaction
        \draw (OA2.-) |- ++(0,1.2) coordinate(top2);
        \draw (top2) to[R, l=$R2$] ++(2.5,0) coordinate(top2end);
        \draw (top2end) |- (OA2.out);
        
        % Masse sur l'entrée +
        \draw (OA2.+) -- ++(0,-0.4) node[ground]{};
        
        % Sortie finale
        \draw (OA2.out) to[short, -o] ++(0.8,0);
        \draw (OA2.out) ++(0.8,0) node[right]{$V_s$};
        
        % Tension d'entrée
        \draw (0,0) node[left]{$V_e$};
        
    \end{circuitikz}
    \caption{Schéma d'un correcteur PI avec des amplificateurs opérationnels}
    \label{fig:schema_pi_aop}
\end{figure}

\subsection{Calcul de la fonction de transfert globale}

La fonction de transfert globale du système s'exprime comme suit : 
\begin{equation*} 
    H(s) = \frac{V_s(s)}{V_e(s)} = \frac{V_a(s)}{V_e(s)} \cdot \frac{V_s(s)}{V_a(s)} = H_1(s) \cdot H_2(s) 
\end{equation*}

Dans cette configuration, on suppose que les amplificateurs opérationnels sont idéaux, ce qui implique $\varepsilon = 0 \Longrightarrow V^+ = V^-$. Le dimensionnement des correcteurs s'effectue en séparant les deux fonctions de transfert.

\subsection{Dimensionnement du correcteur $H_1(s)$}

Pour le premier étage du correcteur, on introduit une impédance équivalente :
\begin{equation*}
    Z_{eq}(s) = R + \frac{1}{C \cdot s} = \frac{RC \cdot s + 1}{sC}
\end{equation*}

En appliquant le théorème de Millman à l'entrée négative du premier AOP, on obtient :
\begin{equation*}
    V^- = \frac{\frac{V_e}{R} + \frac{V_a}{Z_{eq}(s)}}{\frac{1}{R} + \frac{1}{Z_{eq}(s)}}
\end{equation*}

Puisque $V^- = 0V$, le numérateur doit être nul. On a donc :

\begin{equation*}
    \frac{V_e}{R} + \frac{V_a}{Z_{eq}(s)} = 0
\end{equation*}
En réarrangeant, on obtient :
$$\frac{V_a}{Z_{eq}(s)} = - \frac{V_e}{R}$$
$$V_a = - \frac{Z_{eq}(s)}{R} V_e$$
Et donc :
$$H_1(s) = \frac{V_a(s)}{V_e(s)} = - \frac{Z_{eq}(s)}{R} = - \frac{\frac{RC \cdot s+1}{C \cdot s}}{R} = - \frac{RC \cdot s +1}{RC \cdot s}$$

\subsection{Dimensionnement du correcteur $H_2(s)$}

On applique le théorème de Millman au nœud $V^-$ du second AOP :
\begin{equation*}
    V^- = \frac{\frac{V_a(s)}{R_1} + \frac{V_s(s)}{R_2}}{\frac{1}{R_1} + \frac{1}{R_2}}
\end{equation*}
Puisque $V^- = 0V$ (masse virtuelle), le numérateur doit être nul :
\begin{equation*}
    \frac{V_a(s)}{R_1} + \frac{V_s(s)}{R_2} = 0
\end{equation*}
On isole $V_s(s)$ pour trouver la fonction de transfert :
\begin{equation*}
    \frac{V_s(s)}{R_2} = - \frac{V_a(s)}{R_1} \Longrightarrow V_s(s) = - V_a(s) \cdot \frac{R_2}{R_1}
\end{equation*}
La fonction de transfert du second étage s'exprime alors par :
\begin{equation*}
    H_2(s) = \frac{V_s(s)}{V_a(s)} = - \frac{R_2}{R_1}
\end{equation*}

\subsection{Fonction de transfert globale et identification}
On calcule la fonction de transfert globale $H(s)$ en multipliant $H_1(s)$ et $H_2(s)$ :
\begin{equation*}
    H(s) = H_1(s) \cdot H_2(s) = \left( - \frac{RC \cdot s + 1}{RC \cdot s} \right) \cdot \left( - \frac{R_2}{R_1} \right)
\end{equation*}
Les deux signes négatifs s'annulent. On réarrange les termes :
\begin{equation*}
    H(s) = \frac{R_2}{R_1} \cdot \left( \frac{RC \cdot s + 1}{RC \cdot s} \right)
\end{equation*}
On sépare la fraction pour faire apparaître la forme canonique :
\begin{equation*}
    H(s) = \frac{R_2}{R_1} \cdot \left( \frac{RC \cdot s}{RC \cdot s} + \frac{1}{RC \cdot s} \right) = \frac{R_2}{R_1} \left( 1 + \frac{1}{RC \cdot s} \right)
\end{equation*}
La forme de Laplace standard (ou canonique) pour un correcteur PI est :
\begin{equation*}
    H_{PI}(s) = K_p \left( 1 + \frac{1}{T_i s} \right)
\end{equation*}

\subsection{Identification des paramètres}

Par identification directe avec la structure PI souhaitée, on obtient les relations suivantes :

\begin{tcolorbox}[colback=conclusioncolor!5!white, colframe=conclusioncolor!75!black, boxrule=0.5pt, arc=2mm]
\begin{align*}
    K_{p} &= \frac{R_2}{R_1} \\
    T_i &= RC
\end{align*}
\end{tcolorbox}


\subsection{Choix des composants}

À partir des paramètres théoriques des correcteurs PI déterminés précédemment, nous procédons à l'identification des composants passifs normalisés (résistances et condensateurs) dont les valeurs commerciales permettent d'approcher au mieux les caractéristiques désirées. Les résultats de cette identification sont présentés ci-après :

\vspace{0.5cm}

\textbf{Correcteur PI pour l'asservissement en courant :}

Les paramètres du correcteur sont : $K_p = 5{,}053$ et $T_i = K_p / K_i = 1{,}39 \times 10^{-4}$ s.

Par identification avec les relations $K_p = R_2/R_1$ et $T_i = RC$, nous obtenons :
\begin{equation*}
    R \cdot C = 10 \times 10 \times 10^{-6} = 1 \times 10^{-4} \approx T_i \quad \text{et} \quad \frac{R_2}{R_1} = \frac{8{,}2 \times 10^3}{1{,}5 \times 10^3} = 5{,}467 \approx K_p
\end{equation*}

\begin{center}
\renewcommand{\arraystretch}{1.1}
\begin{tabular}{|l|l|}
\hline
Composant & Valeur \\
\hline
$R$ & $10\,\Omega$ \\
\hline
$C$ & $10\,\mu\text{F}$ \\
\hline
$R_1$ & $1{,}5\,\text{k}\Omega$ \\
\hline
$R_2$ & $8{,}2\,\text{k}\Omega$ \\
\hline
\end{tabular}
\end{center}

\vspace{0.5cm}

\textbf{Correcteur PI pour l'asservissement en vitesse :}

Les paramètres du correcteur sont : $K_p = 6{,}8$ et $T_i = K_p / K_i = 13{,}45$ ms.

Par identification avec les relations $K_p = R_2/R_1$ et $T_i = RC$, nous obtenons :
\begin{equation*}
    R \cdot C = 3{,}3 \times 10^3 \times 3{,}9 \times 10^{-6} = 0{,}0129 \approx T_i \quad \text{et} \quad \frac{R_2}{R_1} = \frac{6{,}8 \times 10^3}{1 \times 10^3} = 6{,}8 \approx K_p
\end{equation*}

\begin{center}
\renewcommand{\arraystretch}{1.1}
\begin{tabular}{|l|l|}
\hline
Composant & Valeur \\
\hline
$R$ & $3{,}3\,\text{k}\Omega$ \\
\hline
$C$ & $3{,}9\,\mu\text{F}$ \\
\hline
$R_1$ & $1\,\text{k}\Omega$ \\
\hline
$R_2$ & $6{,}8\,\text{k}\Omega$ \\
\hline
\end{tabular}
\end{center}

\section{Dimensionnement des autres composants de la chaîne d'asservissement}

Après avoir dimensionné les correcteurs PI et les soustracteurs, il est nécessaire de compléter la chaîne d'asservissement avec les autres éléments essentiels : le limiteur de courant pour protéger le système, et les capteurs de vitesse (dynamo tachymétrique et codeur incrémental) qui nécessitent des circuits de conditionnement pour adapter leurs signaux.

\subsection{Limiteur de courant}

Afin de garantir le respect des contraintes fixées par le cahier des charges, il est impératif de limiter le courant circulant dans le système à une valeur maximale de $5\ \text{A}$. Cette protection est assurée par un circuit écrêteur de tension, réalisé à l'aide de diodes et de ponts diviseurs de tension résistifs.

\noindent
\begin{minipage}[t]{0.58\textwidth}
\vspace{0pt}

\subsubsection{Principe de fonctionnement}

Le montage proposé repose sur une architecture de redressement simple alternance permettant d'effectuer un écrêtage asymétrique du signal.

\vspace{0.5cm}

Durant l'alternance positive du signal d'entrée, la diode D se trouve polarisée en direct et devient conductrice. Le courant traverse alors le diviseur résistif constitué des résistances en série ($20\,\text{k}\Omega + 10\,\text{k}\Omega$), générant une chute de tension aux bornes de cet ensemble. Cette tension présente une forme identique à celle de l'alternance positive du signal d'entrée.

\end{minipage}
\hfill
\begin{minipage}[t]{0.38\textwidth}
\vspace{-30pt}
\begin{figure}[H]
    \centering
    \includegraphics[width=\textwidth]{images/Choix_des_composants/Limiteur_Courant.png}
    \caption{\footnotesize{Montage limiteur de tension à diodes}}
    \label{fig:limiteur_tension}
\end{figure}
\end{minipage}

\vspace{0.5cm}

Lors de l'alternance négative, la diode D se trouve polarisée en inverse, ce qui se traduit par une impédance interne très élevée. Dans cette configuration, le courant traversant le diviseur résistif devient quasi nul, entraînant une chute de tension négligeable à ses bornes. Le signal de sortie est ainsi écrêté pour les valeurs négatives.

\vspace{0.5cm}

L'amplificateur opérationnel monté en configuration suiveur, placé en sortie du circuit, assure une fonction d'adaptation d'impédance. Il permet de découpler les étages tout en préservant l'intégrité du signal écrêté sans charge supplémentaire sur le montage à diode.

\subsection{Capteur dynamo tachymétrique}

La dynamo tachymétrique est un capteur analogique qui convertit directement la vitesse de rotation en une tension proportionnelle. Ce composant nécessite un conditionnement de signal pour adapter sa sortie aux niveaux requis par notre système d'asservissement.

\subsubsection{Caractéristiques du capteur}

Le capteur dynamo tachymétrique utilisé possède une constante de conversion $K_{\text{tachy}} = 6 \times 10^{-3}$ V/(rad/s). Cette valeur indique la tension générée par le capteur pour une vitesse de rotation donnée.

\subsubsection{Conditionnement du signal}

Sur PSIM, nous modélisons le capteur en appliquant le facteur $6/1000$ à la vitesse mesurée pour représenter la dynamo tachymétrique réelle. Pour adapter la tension de sortie aux niveaux requis par le système, nous ajoutons un pont diviseur de tension qui divise par deux la tension de sortie.

\begin{figure}[H]
    \centering
    \includegraphics[width=1\textwidth]{images/Choix_des_composants/Tachy/Schéma_PSIM_Choix_Des_Composants_Tachy.png}
    \caption{Schéma de modélisation du capteur tachymétrique sur PSIM}
    \label{fig:schema_tachy}
\end{figure}
\figref{Figure \ref{fig:schema_tachy}}{Modélisation\_Finale\_VALIDE\_13\_01\_2026/06\_Choix\_Des\_Composants/Tachy/Choix\_composants\_Tachy\_VALIDE\_08\_11\_25.psimsch}

\begin{figure}[H]
    \centering
    \includegraphics[width=1\textwidth]{images/Choix_des_composants/Tachy/Courbe_Comparaison_Choix_des_Composants_Tachy_10_11_2025.png}
    \caption{Comparaison entre Simulink et PSIM pour le capteur tachymétrique et les choix de composants}
    \label{fig:comparaison_tachy}
\end{figure}
\figref{Figure \ref{fig:comparaison_tachy}}{Modélisation\_Finale\_VALIDE\_13\_01\_2026/06\_Choix\_Des\_Composants/Tachy/Comparaison\_Composants\_Tachy\_VALIDE\_08\_11\_2025.m}

La figure \ref{fig:comparaison_tachy} illustre la comparaison entre les modélisations Simulink avec correcteurs PI idéaux et PSIM avec composants réels (résistances et condensateurs). On observe une excellente concordance entre les deux courbes, sachant que nous avons à nouveau appliqué un facteur $0{,}99$ aux données Simulink. Le graphe confirme que les choix de composants permettent de reproduire les performances théoriques. Les performances dynamiques restent conformes aux spécifications, démontrant que le passage aux composants réels n'altère pas significativement le comportement de l'asservissement.

\subsection{Codeur incrémental}

Le codeur simulé sous PSIM délivre une tension comprise entre $0$ et $1$ V, alors que le codeur réel fournit un signal de $0$ à $15$ V. Pour corriger cette différence d'échelle et obtenir une simulation représentative, un gain de 15 a été ajouté à l'entrée du bloc de calcul de la vitesse de rotation.

\subsubsection{Circuit de conversion impulsions/tension}

Pour utiliser le codeur incrémental dans notre système d'asservissement analogique, il est nécessaire de convertir les impulsions numériques en une tension analogique proportionnelle à la vitesse. Cette conversion peut être réalisée par un circuit fréquence-tension (convertisseur F/V). Le schéma complet de l'asservissement en vitesse avec codeur incrémental devient ainsi :

\begin{figure}[H]
    \centering
    \includegraphics[width=1\textwidth]{images/Choix_des_composants/Codeur_Incrémental/Schéma_PSIM_Codeur_Incremental_Complet.png}
    \caption{Schéma complet de l'asservissement en vitesse avec codeur incrémental}
    \label{fig:schemame_asservissement_vitesse_codeur_incremental}
\end{figure}
\figref{Figure \ref{fig:schemame_asservissement_vitesse_codeur_incremental}}{Modélisation\_Finale\_VALIDE\_13\_01\_2026/06\_Choix\_Des\_Composants/Codeur/Choix\_Composants\_Codeur\_Incremental\_VALIDE\_08\_11\_25.psimsch}
\newpage
\begin{figure}[H]
    \centering
    \includegraphics[width=1\textwidth]{images/Choix_des_composants/Codeur_Incrémental/Courbe_Choix_des_Composants_Codeur_Incremental_10_11_2025.png}
    \caption{Comparaison entre Simulink et PSIM pour le codeur incrémental et les choix de composants}
    \label{fig:comparaison_codeur_incremental}
\end{figure}
\figref{Figure \ref{fig:comparaison_codeur_incremental}}{Modélisation\_Finale\_VALIDE\_13\_01\_2026/06\_Choix\_Des\_Composants/Codeur/Comparaison\_Composants\_Codeur\_Incremental\_VALIDE\_08\_11\_25.m}

\begin{figure}[H]
    \centering
    \includegraphics[width=0.8\textwidth]{images/Choix_des_composants/Tachy/Zoom_Courbe_Comparaison_Choix_des_Composants_Tachy_10_11_2025.png}
    \caption{Zoom sur la comparaison entre Simulink et PSIM pour le capteur tachymétrique}
    \label{fig:zoom_comparaison_tachy}
\end{figure}


La figure \ref{fig:comparaison_codeur_incremental} illustre la comparaison entre les modélisations Simulink avec correcteurs PI idéaux et PSIM avec composants réels (résistances et condensateurs). On observe une bonne concordance entre les deux courbes, confirmant que les choix de composants permettent de reproduire les performances théoriques. Les oscillations caractéristiques du codeur incrémental sont présentes sur la simulation PSIM, validant la cohérence de la modélisation du capteur numérique. Les performances dynamiques restent conformes aux spécifications, démontrant que le passage aux composants réels n'altère pas significativement le comportement de l'asservissement.