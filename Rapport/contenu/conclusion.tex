\chapter*{Conclusion}
\addcontentsline{toc}{chapter}{Conclusion}

Ce projet d'asservissement d'une machine à courant continu a permis d'acquérir une vision complète de la chaîne de régulation, depuis la modélisation théorique jusqu'au dimensionnement des composants électroniques réels. À ce stade du projet, il convient de dresser un bilan des travaux réalisés et d'identifier les tâches qui restent à accomplir pour finaliser le système.

\section*{Travaux restant à réaliser}

Malgré les avancées significatives effectuées en simulation et en dimensionnement, plusieurs étapes essentielles restent à compléter avant la finalisation du projet :

\begin{enumerate}
    \item \textbf{Réalisation du circuit imprimé (PCB)} : La conception et la fabrication de la carte électronique permettant d'implémenter physiquement l'ensemble des circuits dimensionnés (correcteurs PI, soustracteurs, limiteur de tension, conditionnement des capteurs).
    
    \item \textbf{Assemblage et câblage du système} : Le montage des composants sur le PCB, le raccordement du moteur, des capteurs (dynamo tachymétrique et codeur incrémental) et de l'alimentation de puissance.
    
    \item \textbf{Tests et validation expérimentale} : La mise en service progressive du système réel, comprenant :
    \begin{itemize}
        \item La vérification du bon fonctionnement des correcteurs PI analogiques
        \item Les tests de l'asservissement en courant puis en vitesse
        \item La comparaison des performances mesurées avec les résultats de simulation
    \end{itemize}
    
    \item \textbf{Réglage et optimisation} : L'ajustement fin des paramètres si nécessaire pour compenser les écarts entre le modèle théorique et la réalité (tolérances des composants, pertes non modélisées, perturbations électromagnétiques).
    
    \item \textbf{Caractérisation complète} : L'évaluation des performances dans différentes conditions de fonctionnement (variations de charge, changements de consigne, robustesse face aux perturbations).
\end{enumerate}

\section*{Bilan et perspectives}

Les travaux menés ont établi des fondations solides pour la réalisation pratique du système. La modélisation mathématique de la MCC a été développée et validée sur MATLAB/Simulink et PSIM, incluant la machine à vide, avec charge mécanique et hacheur quatre quadrants. Deux boucles de régulation en cascade ont été conçues et optimisées, respectant les spécifications du cahier des charges en termes de temps de réponse et de dépassement. L'étude comparative de deux capteurs (dynamo tachymétrique et codeur incrémental) a validé la flexibilité du système, et le dimensionnement complet des circuits analogiques réels a été réalisé avec succès.

\vspace{0.5cm}

Ce projet a permis de développer une méthodologie rigoureuse allant de la modélisation théorique à la réalisation pratique, tout en maîtrisant les outils de simulation et en validant systématiquement chaque étape par comparaison croisée. La réalisation expérimentale du système constituera l'aboutissement de ces travaux, permettant de confronter les prédictions théoriques à la réalité. Les fondations méthodologiques acquises constitueront des atouts précieux pour les développements futurs et l'adaptation du système à d'autres applications industrielles.
