\chapter*{Conclusion}
\addcontentsline{toc}{chapter}{Conclusion}

Ce projet d'asservissement de vitesse d'une machine à courant continu nous a permis de mettre en application l'ensemble des connaissances acquises en automatique et en électronique de puissance. L'objectif était de concevoir, réaliser et valider une carte de commande analogique assurant la régulation de vitesse via deux boucles imbriquées : une boucle de courant interne pour la protection et le contrôle du couple, et une boucle de vitesse externe pour la précision de la réponse.

Au terme de ce travail, nous pouvons affirmer que l'objectif est atteint. La démarche adoptée, allant de la modélisation théorique sous Matlab et PSIM jusqu'à la mise en œuvre pratique sur circuit imprimé (PCB), s'est révélée fructueuse.

La phase de simulation a été cruciale pour dimensionner correctement les correcteurs proportionnels intégraux (PI) et valider la stratégie de commande avant tout assemblage physique. Elle nous a permis de prédire le comportement du système et de fixer les valeurs des composants passifs.

La phase de réalisation et de tests sur maquette a été l'occasion de confronter la théorie à la réalité. Nous avons dû faire face à des imprévus techniques, notamment un mauvais positionnement du limiteur de courant et une erreur de valeur de capacité sur le correcteur de courant. L'identification et la correction de ces erreurs ont constitué une part importante de l'apprentissage, démontrant l'importance des phases de validation par étapes (tests sur plaquette puis sur maquette).

Une fois les rectifications apportées, les essais finaux ont montré une très bonne corrélation entre les simulations et le comportement réel du moteur. Que ce soit avec la dynamo tachymétrique ou le codeur incrémental, les performances dynamiques mesurées sont conformes au cahier des charges :
\begin{itemize}
    \item Un temps de réponse d'environ 100 ms.
    \item Un dépassement maîtrisé autour de 16\%.
    \item Une limitation efficace du courant à 5 A, assurant la protection de la machine.
\end{itemize}

En conclusion, ce projet a été une expérience complète d'ingénierie. Il nous a permis de maîtriser la chaîne de conception d'un système asservi industriel et de comprendre les interactions fines entre la commande (automatique) et la puissance (électronique). La carte réalisée est fonctionnelle, robuste et respecte les contraintes de sécurité et de performance imposées.
