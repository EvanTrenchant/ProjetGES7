\chapter{Cahier des charges}

\section{Notations des variables pour le projet}

Le tableau suivant récapitule l'ensemble des notations utilisées dans ce projet, ainsi que les valeurs et unités correspondantes pour notre moteur à courant continu.

\begin{table}[H]
\centering
\caption{Notations des variables pour le projet}
\label{tab:notations}
\begin{tabular}{|c|p{5.5cm}|c|c|}
\hline
\textbf{Notations} & \textbf{Nom} & \textbf{Valeurs} & \textbf{Unités} \\
\hline
\multicolumn{4}{|c|}{\textbf{Paramètres nominaux}} \\
\hline
$M_0$ & couple en rotation lente & 0,54 & N.m \\
\hline
$I_0$ & courant permanent rotation lente & 4,5 & A \\
\hline
$U$ & tension d'alimentation de définition & 49 & V \\
\hline
$N$ & vitesse de définition & 3000 & tr/min \\
\hline
\multicolumn{4}{|c|}{\textbf{Paramètres maximaux}} \\
\hline
$U_{max}$ & tension maximale & 65 & V \\
\hline
$N_{max}$ & vitesse maximale & 4800 & tr/min \\
\hline
$I_{max}$ & courant impulsionnel & 13 & A \\
\hline
\multicolumn{4}{|c|}{\textbf{Constantes électromécaniques}} \\
\hline
$K_e$ & constante de fem (à 25°C) & 13,3 & V/(1000 tr/min) \\
\hline
$K_\Phi$ & constante de couple électromagnétique & 0,127 & N.m/A \\
\hline
\multicolumn{4}{|c|}{\textbf{Paramètres de frottement}} \\
\hline
$T_f$ & couple de frottement sec & 2,4 & N.cm \\
\hline
$K_d$ & coefficient de viscosité & 0,53 & N.cm/(1000 tr/min) \\
\hline
\multicolumn{4}{|c|}{\textbf{Paramètres électriques}} \\
\hline
$R$ & résistance du bobinage (à 25°C) & 1,52 & $\Omega$ \\
\hline
$L$ & inductance du bobinage & 2,2 & mH \\
\hline
$R_{ch}$ & résistance de charge & 10 & $\Omega$ \\
\hline
\multicolumn{4}{|c|}{\textbf{Paramètres mécaniques}} \\
\hline
$J$ & inertie du rotor & $8{,}3 \cdot 10^{-5}$ & kg.m$^2$ \\
\hline
$M$ & masse du moteur & 1,34 & kg \\
\hline
$T_{th}$ & constante de temps thermique & 7 & min \\
\hline
\multicolumn{4}{|c|}{\textbf{Variables dynamiques}} \\
\hline
$\Omega$ & vitesse de rotation & & rad/s \\
\hline
$C_m$ & couple électromagnétique & & N.m \\
\hline
$C_r$ & couple résistant total & & N.m \\
\hline
$i$ & courant d'induit & & A \\
\hline
$u$ & tension d'alimentation & & V \\
\hline
$e$ & force contre-électromotrice & & V \\
\hline
\end{tabular}
\end{table}

\section{Spécifications du système}

Le cahier des charges du projet est défini comme suit :
\section*{Asservissement en courant}
\begin{itemize}
    \item Temps de réponse maximal de 10 fois la periode de la MLI soit 0,45 ms
    \item Depassement maximal de 20\%
\end{itemize}

\section*{Asservissement en vitesse}
\begin{itemize}
    \item Depassement maximal de 20\%
\end{itemize}
