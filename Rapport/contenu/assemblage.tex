\chapter{Assemblage du système}

\section{Commande des composants}

En parallèle de la réalisation du circuit imprimé, nous avons réalisé la commande de composants chez le fournisseur Farnell.

 Le budget étant restreint à 25\,\euro, nous avons fait en sorte de récupérer un maximum de composants déjà présents en PFGE pour limiter le coût de notre commande. De plus, nous avons mutualisé les commandes des différents groupes pour éviter les frais de livraison et diminuer le coût de certains composants achetés en plus gros lots.

Voici la liste des composants que nous avons dû commander :

\begin{tableau}[H]
    \centering
    \begin{tabular}{|l|l|c|c|c|}
        \hline
        \textbf{Composant} & \textbf{Référence} & \textbf{Lien Farnell} & \textbf{Quantité} & \textbf{Prix HT} \\
        \hline
        TL082 & TL082CDT & \href{https://fr.farnell.com/stmicroelectronics/tl082cdt/amplificateur-op-4mhz-0-a-85-c/dp/3367325}{Lien} & 6 & 3,52\,\euro \\
        \hline
        Diodes & 1N4148WS & \href{https://fr.farnell.com/onsemi/1n4148ws/diode-petit-signal-75v-1-5a-sod/dp/2453269}{Lien} & 3 & 0,50\,\euro \\
        \hline
        Potentiomètre 10k$\Omega$ & T93YA103KT20 & \href{https://fr.farnell.com/vishay/t93ya103kt20/potentiometre-trimmer-10k-23-tours/dp/1141404}{Lien} & 2 & 1,90\,\euro \\
        \hline
        Bascule D & HEF40175BT,653 & \href{https://fr.farnell.com/nexperia/hef40175bt-653/bascule-d-40-a-85-c/dp/3442180}{Lien} & 1 & 1,06\,\euro \\
        \hline
        MUX & DG419DY-E3 & \href{https://fr.farnell.com/vishay/dg419dy-e3/ic-mux-precision-cmos-smd/dp/1469446}{Lien} & 1 & 1,57\,\euro \\
        \hline
        Capacité 470 $\mu$F & 25ZLK470M8X20 & \href{https://fr.farnell.com/rubycon/25zlk470m8x20/condensateur-470-f-25v-20/dp/1831286}{Lien} & 2 & 0,83\,\euro \\
        \hline
        Résistance 47k$\Omega$ & CRCW080547K0FKEA & \href{https://fr.farnell.com/vishay/crcw080547k0fkea/res-couche-epaisse-47k-1-0-125w/dp/1469929}{Lien} & 1 & 0,22\,\euro \\
        \hline
        Résistance 8,2k$\Omega$ & MCWF08P8201FTL & \href{https://fr.farnell.com/multicomp-pro/mcwf08p8201ftl/res-couche-epaisse-8-2k-1-0-25w/dp/2694155}{Lien} & 1 & 0,11\,\euro \\
        \hline
        Résistance 560k$\Omega$ & ESR10EZPF5603 & \href{https://fr.farnell.com/rohm/esr10ezpf5603/res-560k-1-0-4w-couche-paisse/dp/4009519}{Lien} & 1 & 0,75\,\euro \\
        \hline
        Capacité 560 nF & MC1206B564K250CT & \href{https://fr.farnell.com/multicomp-pro/mc1206b564k250ct/cond-0-56-f-25v-10-x7r-1206/dp/1759323}{Lien} & 1 & 0,79\,\euro \\
        \hline
        Capacité 6.8 nF & MC1206B682K201CT & \href{https://fr.farnell.com/multicomp-pro/mc1206b682k201ct/cond-6800pf-200v-10-x7r-1206/dp/1855871}{Lien} & 1 & 0,41\,\euro \\
        \hline
        Capacité 1.2 nF & CC1206JKNPOZBN122 & \href{https://fr.farnell.com/yageo/cc1206jknpozbn122/cond-1200pf-630v-mlcc-1206/dp/4166976}{Lien} & 1 & 0,63\,\euro \\
        \hline
        \hline
        \textbf{TOTAL TTC} & & & & \textbf{14,80\,\euro} \\
        \hline
    \end{tabular}
    \caption{Liste des composants commandés}
    \label{tab:composants}
\end{tableau}

\section{Soudure du PCB}

\noindent
\begin{minipage}[t]{0.5\textwidth}
\vspace{0pt}
Une fois que le PCB a été produit par les techniciens et que nous avons reçu la commande, nous sommes passés à la soudure de la carte.

Pour ce faire, nous avions décidé de réaliser une carte entièrement avec des composants montés en surface pour obtenir la carte la plus petite possible. Nous avons donc réalisé en premier la soudure des composants CMS au four à souder, nous avons donc placé les AOP, résistances, capacités et diodes ainsi que la bascule D et l'interrupteur commandé.

Une fois ces composants soudés, nous nous sommes occupés des composants traversant : les différents borniers, potentiomètres et capacités de découplage. Nous avons soudé ces éléments au fer à souder.

Cela nous donne le PCB visible ci-contre.
\end{minipage}
\hfill
\begin{minipage}[t]{0.48\textwidth}
\vspace{-5pt}
\begin{figure}[H]
    \centering
    \includegraphics[width=\textwidth]{images/Image PCB/image3.jpg}
    \caption{PCB après soudure des composants}
    \label{fig:pcb_soudure}
\end{figure}
\end{minipage}

Pour ce qui est de la connexion de notre carte au système réel, nous devons utiliser des câbles coaxiaux pour récupérer les différentes tensions utiles au fonctionnement du système. De plus, notre système doit être alimenté en +15 V et -15 V à l'aide d'une alimentation de laboratoire.

Enfin, il doit être possible de choisir avec quel capteur nous réalisons l'asservissement, donc nous utilisons un interrupteur pour basculer entre la dynamo tachymétrique et le codeur incrémental qui est relié à un port VGA.

Ce qui fait que nous devons réaliser la soudure de fils reliant la carte à ces composants de connectique :
\begin{itemize}
    \item 4 fiches BNC
    \item 3 fiches bananes
    \item 1 interrupteur
    \item 1 connecteur VGA
\end{itemize}



\section{Réalisation du boîtier}

Pour maintenir le PCB et les connectiques en place, nous avons réalisé une boîte imprimée en 3D pour y fixer tous les composants.

Cette boîte a été pensée pour être la plus pratique possible si des modifications doivent être réalisées. C'est pourquoi nous avons décidé de fixer la totalité des composants sous le couvercle de cette boîte pour que, si nous voulons l'ouvrir, nous ayons facilement accès à tous les composants. Le but était d'éviter que des fils relient le couvercle et la boîte, ce qui complique l'ouverture.

Nous avons aussi fait en sorte d'indiquer sur le couvercle l'utilité de chaque connectique en imprimant le texte d'une autre couleur.

\noindent
\begin{minipage}[t]{0.40\textwidth}
\vspace{0pt}
\begin{figure}[H]
    \centering
    \includegraphics[width=\textwidth]{images/Modelisation3d/Vue3D.png}
    \caption{Vue 3D du boîtier}
    \label{fig:boitier_3d}
\end{figure}
\end{minipage}
\hfill
\begin{minipage}[t]{0.55\textwidth}
\vspace{0pt}
\begin{figure}[H]
    \centering
    \includegraphics[width=\textwidth]{images/Modelisation3d/Vue3DOuvert.png}
    \caption{Vue 3D du boîtier ouvert}
    \label{fig:boitier_3d_ouvert}
\end{figure}
\end{minipage}

\vspace{0.5cm}

Nous avons mis en place plusieurs éléments visant à rendre notre boîte plus fonctionnelle :
\begin{itemize}
    \item Des empreintes à double meplat pour fixer les fiches BNC, évitant ainsi qu'elles ne tournent lors des connexions/déconnexions. Cela permet également d'aligner parfaitement les fiches BNC.
    \item Un épaulement sur le pourtour du couvercle pour qu'il s'emboîte parfaitement dans la boîte, même sans être vissé, et ainsi éviter toute entrée de poussière.
    \item Des pieds en caoutchouc pour éviter que la boîte ne glisse sur le poste de travail.
\end{itemize}

\noindent
\begin{minipage}[t]{0.30\textwidth}
\vspace{0pt}
\begin{figure}[H]
    \centering
    \includegraphics[width=\textwidth]{images/Modelisation3d/Meplat.png}
    \caption{Empreintes à double meplat}
    \label{fig:meplat}
\end{figure}
\end{minipage}
\hfill
\begin{minipage}[t]{0.30\textwidth}
\vspace{0pt}
\begin{figure}[H]
    \centering
    \includegraphics[width=\textwidth]{images/Modelisation3d/epaulement.png}
    \caption{Épaulement du couvercle}
    \label{fig:epaulement}
\end{figure}
\end{minipage}
\hfill
\begin{minipage}[t]{0.30\textwidth}
\vspace{0pt}
\begin{figure}[H]
    \centering
    \includegraphics[width=\textwidth]{images/Modelisation3d/Patin.png}
    \caption{Pieds en caoutchouc}
    \label{fig:patin}
\end{figure}
\end{minipage}

\vspace{10pt}

Cette modélisation 3D complète de notre système nous a permis de vérifier que tous les composants rentraient correctement dans la boîte avant de lancer l'impression 3D.

Voici le résultat du projet entièrement assemblé :

\noindent
\begin{minipage}[t]{0.45\textwidth}
\vspace{0pt}
\begin{figure}[H]
    \centering
    \includegraphics[width=\textwidth]{images/Image PCB/image2.jpg}
    \caption{Couvercle du boîtier}
    \label{fig:boitier_couvercle}
\end{figure}
\end{minipage}
\hfill
\begin{minipage}[t]{0.53\textwidth}
\vspace{0pt}
\begin{figure}[H]
    \centering
    \includegraphics[width=\textwidth]{images/Image PCB/image4.jpg}
    \caption{Vue d'ensemble du boîtier}
    \label{fig:boitier_complet}
\end{figure}
\end{minipage}

\vspace{10pt}

Pour réaliser les connexions entre le PCB et les différentes connectiques, nous avons utilisé des câbles que nous avons soudés aux fiches BNC, puis vissés aux borniers du PCB. Les fiches bananes ont été connectées au PCB via des câbles avec des cosses pour effectuer les connexions.
\\
Pour faire cela proprement, nous avons coupé les fils pour qu'ils soient les plus courts possible afin d'éviter leur encombrement. Nous avons également ajouté des gaines thermo-rétractables à chaque soudure pour éviter tout faux contact ou court-circuit.
\\
Voici le PCB relié à tous les organes de connexion :

\begin{figure}[H]
    \centering
    \includegraphics[width=0.8\textwidth]{images/Image PCB/image1.jpg}
    \caption{PCB relié à toutes les connectiques}
    \label{fig:pcb_connectique}
\end{figure}
