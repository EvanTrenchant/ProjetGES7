\chapter{Planification du projet}

La réalisation de ce projet d'asservissement d'une machine à courant continu nécessite une planification rigoureuse pour garantir l'atteinte des objectifs dans les délais impartis. Cette section présente l'organisation temporelle du projet et le diagramme de Gantt détaillant les différentes phases de développement.

\section*{Diagramme de Gantt}

Le diagramme de Gantt ci-dessous illustre la planification détaillée du projet sur 9 semaines, avec les dépendances entre les tâches et les jalons importants.

% Original Gantt chart (14 semaines) conservé en commentaire ci-dessous :
%
%\begin{figure}[H]
%    \centering
%    \begin{ganttchart}[
%        vgrid,
%        hgrid,
%        x unit=0.8cm,
%        y unit chart=0.7cm,
%        title height=1,
%        title label font=\footnotesize,
%        bar label font=\footnotesize,
%        group label font=\footnotesize,
%        milestone label font=\footnotesize,
%        bar height=0.6,
%        group height=0.6,
%        milestone height=0.6,
%        bar/.append style={fill=INSAbleu!70},
%        group/.append style={fill=INSAbleu!50},
%        milestone/.append style={fill=red!70, rounded corners=2pt},
%        progress label text={},
%        link/.style={->, thick}
%    ]{1}{14}
%        
%        % En-tête du diagramme
%        \gantttitle{Planning Projet GE S7 - Asservissement MCC}{14} \\
%        \gantttitle{Semaine}{14} \\
%        \gantttitlelist{1,...,14}{1} \\
%        
%        % Phase 1 : Modélisation de base
%        \ganttgroup{Phase 1 : Modélisation de base}{1}{3} \\
%        \ganttbar{Étude théorique MCC}{1}{2} \\
%        \ganttbar{Modélisation MATLAB}{2}{3} \\
%        \ganttbar{Modélisation PSIM}{2}{3} \\
%        \ganttmilestone{Validation modèle moteur seul}{3} \\
%        
%        % Phase 2 : Modélisation complète  
%        \ganttgroup{Phase 2 : Modélisation complète}{4}{6} \\
%        \ganttbar{Intégration charge mécanique}{4}{5} \\
%        \ganttbar{Validation croisée MATLAB/PSIM}{5}{6} \\
%        \ganttbar{Tests de robustesse}{6}{6} \\
%        \ganttmilestone{Modèle complet validé}{6} \\
%        
%        % Phase 3 : Boucle de courant
%        \ganttgroup{Phase 3 : Boucle de courant}{7}{9} \\
%        \ganttbar{Conception correcteur courant}{7}{8} \\
%        \ganttbar{Réglage et optimisation}{8}{9} \\
%        \ganttbar{Validation spécifications}{9}{9} \\
%        \ganttmilestone{Asservissement courant opérationnel}{9} \\
%        
%        % Phase 4 : Boucle de vitesse
%        \ganttgroup{Phase 4 : Boucle de vitesse}{10}{12} \\
%        \ganttbar{Conception correcteur vitesse}{10}{11} \\
%        \ganttbar{Intégration boucles imbriquées}{11}{12} \\
%        \ganttbar{Tests capteurs (tachy/codeur)}{11}{12} \\
%        \ganttbar{Validation performance globale}{12}{12} \\
%        \ganttmilestone{Système d'asservissement complet}{12} \\
%        
%        % Phase 5 : Finalisation
%        \ganttgroup{Phase 5 : Validation finale}{13}{14} \\
%        \ganttbar{Optimisation finale}{13}{13} \\
%        \ganttbar{Documentation et rapport}{13}{14} \\
%        \ganttbar{Préparation soutenance}{14}{14} \\
%        \ganttmilestone{Projet finalisé}{14} \\
%        
%        % Liens entre les tâches
%        \ganttlink{elem3}{elem5}
%        \ganttlink{elem7}{elem9}
%        \ganttlink{elem11}{elem13}
%        \ganttlink{elem16}{elem18}
%    \end{ganttchart}
%    \caption{Diagramme de Gantt du projet d'asservissement MCC (version complète, 14 semaines) -- conservé en commentaire}
%    \label{fig:gantt_projet_full}
%\end{figure}

% Version tronquée du diagramme de Gantt (jusqu'à 9 semaines) pour rapport intermédiaire
\begin{figure}[H]
    \centering
    % Encapsuler le diagramme dans une boite redimensionnée pour éviter le débordement
    \resizebox{\linewidth}{!}{%
    \begin{ganttchart}[
        vgrid,
        hgrid,
        x unit=\dimexpr\textwidth/9\relax,
        % Augmenter l'espacement vertical entre les lignes (légendes/tâches)
        y unit chart=0.95cm,
        title height=1,
        title label font=\footnotesize,
        bar label font=\footnotesize,
        group label font=\footnotesize,
        milestone label font=\footnotesize,
        bar height=0.6,
        group height=0.6,
        milestone height=0.6,
        bar/.append style={fill=INSAbleu!70, rounded corners=4pt},
        group/.append style={fill=remindercolor, rounded corners=4pt},
        milestone/.append style={fill=INSArouge, rounded corners=4pt},
        % Espacement entre le texte de label (légende) et les barres
        bar label node/.append style={inner xsep=6pt},
        progress label text={},
        link/.style={->, thick, rounded corners=2pt}
    ]{1}{9}

    % En-tête du diagramme
    \gantttitle{Semaine}{9} \\
    \gantttitlelist{1,...,9}{1} \\

    % Phase 1 : Modélisation de base (1-2)
    \ganttgroup{Phase 1 : Modélisation de base}{1}{2} \\
    \ganttbar{Étude théorique MCC}{1}{1} \\
    \ganttbar{Modélisation MATLAB}{1}{2} \\
    \ganttbar{Modélisation PSIM}{2}{2} \\
    \ganttmilestone{Validation modèle moteur seul}{2} \\

    % Phase 2 : Modélisation avec charge et frottements (3)
    \ganttgroup{Phase 2 : Modélisation charge et frottements}{3}{3} \\
    \ganttbar{Modélisation charge \& frottements}{3}{3} \\
    \ganttmilestone{Modèle avec charge validé}{3} \\

    % Phase 3 : Modélisation complète (incl. étude hacheur) (4-5)
    \ganttgroup{Phase 3 : Modélisation complète}{4}{5} \\
    \ganttbar{Étude du hacheur}{4}{5} \\
    \ganttbar{Validation croisée MATLAB/PSIM}{5}{5} \\
    \ganttmilestone{Modèle complet validé}{5} \\

    % Phase 4 : Asservissements (courant puis vitesse) (5-7)
    \ganttgroup{Phase 4 : Asservissements}{6}{9} \\
    \ganttbar{Conception correcteur courant}{6}{6} \\
    \ganttbar{Réglage et optimisation courant}{6}{6} \\
    \ganttbar{Conception correcteur vitesse}{7}{7} \\
    \ganttbar{Intégration boucles imbriquées}{7}{9} \\
    \ganttmilestone{Système d'asservissement complet}{7} \\

    % Phase 5 : Rédaction et finalisation (8-9)
    \ganttgroup{Phase 5 : Rédaction et finalisation}{8}{9} \\
    \ganttbar{Rédaction du rapport intermédiaire}{8}{9} \\
    \ganttmilestone{Livrable intermédiaire}{9}

    % Liens entre les tâches (simplifiés)
    \ganttlink{elem3}{elem5}
    \ganttlink{elem6}{elem8}
    \ganttlink{elem10}{elem12}
    \ganttlink{elem16}{elem18}
    \end{ganttchart}%
    }% fin resizebox
    \caption{Diagramme de Gantt — récapitulatif jusqu'à 9 semaines}
    \label{fig:gantt_projet_short}
\end{figure}