\chapter{Planification du projet}

La réalisation de ce projet d'asservissement d'une machine à courant continu nécessite une planification rigoureuse pour garantir l'atteinte des objectifs dans les délais impartis. Cette section présente l'organisation temporelle du projet et le diagramme de Gantt détaillant les différentes phases de développement.

\section*{Diagramme de Gantt}

Le diagramme de Gantt ci-dessous illustre la planification détaillée du projet s'étendant de septembre à janvier, avec les dépendances entre les tâches et les jalons importants.


\begin{figure}[H]
    \centering
    % Encapsuler le diagramme pour qu'il prenne toute la largeur disponible
    \resizebox{\linewidth}{!}{%
    \begin{ganttchart}[
        vgrid,
        hgrid,
        expand chart=\linewidth,
        y unit chart=0.35cm,
        title height=1,
        title label font=\footnotesize,
        bar label font=\scriptsize,
        group label font=\scriptsize\bfseries\color{INSArouge},
        milestone label font=\scriptsize,
        bar height=0.5,
        group height=0.5,
        milestone height=0.5,
        bar/.append style={fill=INSAbleu!70, rounded corners=3pt},
        group/.append style={fill=INSArouge!70, rounded corners=3pt},
        milestone/.append style={fill=INSArouge, rounded corners=3pt},
        bar label node/.append style={inner xsep=4pt},
        progress label text={},
        link/.style={->, thick, rounded corners=2pt}
    ]{1}{19}

    % En-tête du diagramme
    \gantttitle{Calendrier 2025 - 2026 (Semaines calendaires)}{19} \\
    \gantttitlelist{37,...,52}{1}
    \gantttitlelist{1,...,3}{1} \\

    % Phase 1 : Modélisation de base (W37-38)
    \ganttgroup{Phase 1 : Modélisation de base}{1}{2} \\
    \ganttbar{Étude théorique MCC}{1}{1} \\
    \ganttbar{Modélisation MATLAB}{1}{2} \\
    \ganttbar{Modélisation PSIM}{2}{2} \\
    \ganttmilestone{Validation modèle moteur seul}{2} \\

    % Phase 2 : Charge (W39)
    \ganttgroup{Phase 2 : Charge méca. \& Frottements}{3}{3} \\
    \ganttbar{Modélisation charge \& frottements}{3}{3} \\
    \ganttmilestone{Modèle avec charge validé}{3} \\

    % Phase 3 : Modèle Complet (W40-41)
    \ganttgroup{Phase 3 : Modélisation complète}{4}{5} \\
    \ganttbar{Étude du hacheur}{4}{5} \\
    \ganttbar{Validation croisée MATLAB/PSIM}{5}{5} \\
    \ganttmilestone{Modèle complet validé}{5} \\

    % Phase 4 : Asservissements (W42-43/44)
    \ganttgroup{Phase 4 : Asservissements}{6}{8} \\
    \ganttbar{Conception correcteur courant}{6}{6} \\
    \ganttbar{Réglage et optimisation courant}{6}{6} \\
    \ganttbar{Conception correcteur vitesse}{7}{7} \\
    \ganttbar{Intégration boucles imbriquées}{7}{8} \\
    \ganttmilestone{Système d'asservissement complet}{8} \\

    % Phase 5 : Rapport Intermédiaire & Dim (W44-46)
    \ganttgroup{Phase 5 : Dim. \& Rapport Int.}{8}{10} \\
    \ganttbar{Dimensionnement Composants \& Tachy}{8}{10} \\
    \ganttbar{Rédaction du rapport intermédiaire}{8}{9} \\
    \ganttmilestone{Livrable intermédiaire (09/11)}{9} \\

    % Phase 6 : PCB (W46-48)
    \ganttgroup{Phase 6 : Réalisation du PCB}{10}{12} \\
    \ganttbar{Conception du circuit imprimé}{10}{12} \\
    \ganttmilestone{Commande des composants (26/11)}{12} \\

    % Phase 7 : Assemblage (W49-50)
    \ganttgroup{Phase 7 : Assemblage \& Boitier 3D}{12}{13} \\
    \ganttbar{Modélisation boitier, ajustements}{12}{13} \\
    \ganttbar{Assemblage du système complet}{12}{13} \\

    % Phase 8 : Tests (W50-W02)
    \ganttgroup{Phase 8 : Tests \& Corrections}{13}{18} \\
    \ganttbar{Premiers essais concluants}{13}{14} \\
    \ganttbar{Mise en place des corrections}{14}{18} \\
    \ganttmilestone{Système final opérationnel}{18} \\

    % Phase 9 : Finalisation (W02-03)
    \ganttgroup{Phase 9 : Finalisation du projet}{18}{19} \\
    \ganttbar{Rédaction Rapport Final \& Soutenance}{18}{19} \\
    \ganttmilestone{Rapport Final et Soutenance}{19}

    % Liens (Mise à jour pour inclure les nouveaux éléments)
    \ganttlink{elem3}{elem5}   % Moteur Seul -> Charge
    \ganttlink{elem6}{elem8}   % Charge -> Modèle Charge Validé
    \ganttlink{elem10}{elem12}  % Validé -> Hacheur
    \ganttlink{elem16}{elem18} % Opt Courant -> Syst Complet
    \ganttlink{elem20}{elem22} % Dim -> PCB
    \ganttlink{elem23}{elem25} % Commande -> Assemblage
    \ganttlink{elem30}{elem32} % Corrections -> Finalisation


    \end{ganttchart}%
    }% fin resizebox
    \caption{Diagramme de Gantt étendu du projet (Septembre - Janvier)}
    \label{fig:gantt_projet_short}
\end{figure}