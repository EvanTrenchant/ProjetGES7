\chapter{Étude de la MCC et de son hacheur}

\section{Modélisation de la MCC à vide}

\subsection{Équations de la MCC}

La machine à courant continu (MCC) est un dispositif électromécanique capable de convertir l'énergie électrique en énergie mécanique, et inversement. Cette conversion bidirectionnelle s'effectue grâce au passage d'un courant continu dans son induit, générant ainsi un couple moteur qui entraîne l'arbre en rotation. En mode générateur, le principe est inversé : la rotation de l'arbre produit une force électromotrice aux bornes de la machine. La MCC constitue ainsi un élément clé dans de nombreux systèmes d'entraînement industriels et de conversion d'énergie.

\subsubsection{Formules d'une MCC sans pertes et sans charge}

Les équations régissant le comportement d'une machine à courant continu sans pertes et sans charge sont les suivantes :

\begin{tcolorbox}[colback=remindercolor!5!white, colframe=remindercolor!75!black, title=Équations temporelles de la MCC]
\begin{equation*}
u(t) = R \cdot i(t) + L \frac{di(t)}{dt} + e(t)
\end{equation*}

\begin{equation*}
C_{em}(t) = K_c \cdot i(t)
\end{equation*}

\begin{equation*}
e(t) = K_e \cdot \Omega(t)
\end{equation*}

\begin{equation*}
J \frac{d\Omega}{dt} = C_{em}(t)
\end{equation*}

\vspace{0.2cm}
\centering
On remplacera ici $K_c$ et  $K_e$  par $K_\Phi$.
\end{tcolorbox}

\subsubsection{Transformée de Laplace}

Les équations précédentes peuvent être exprimées dans le domaine de Laplace, ce qui facilite l'analyse et la modélisation du système :

\begin{tcolorbox}[colback=remindercolor!5!white, colframe=remindercolor!75!black, title=Équations dans le domaine de Laplace]
\begin{equation*}
I(s) = \frac{U(s) - K_\Phi \Omega}{R + L \cdot s}
\end{equation*}

\begin{equation*}
C_{em}(s) = K_\Phi \cdot I(s)
\end{equation*}

\begin{equation*}
E(s) = K_\Phi \cdot \Omega(s)
\end{equation*}

\begin{equation*}
\Omega(s) = \frac{K_\Phi \cdot I(s)}{J \cdot s}
\end{equation*}
\end{tcolorbox}

\subsubsection{Conversion vitesse-tension}

Le moteur possède une force contre-électromotrice $E$ proportionnelle à sa vitesse de rotation. 
Les données constructeur indiquent : $E=13{,}3 \mathrm{~V}$ pour $1000 \mathrm{~tr/min}$.

En notant $N$ la vitesse en tr/min et $\Omega$ la vitesse en rad/s, on obtient les relations suivantes :

\begin{equation*}
\begin{aligned}
& E=\frac{13,3}{1\,000} \cdot N \\
& \Omega=\frac{\pi N}{30} \Rightarrow E=\underbrace{\frac{13,3}{1\,000} \cdot \frac{30}{\pi}}_{0,127} \cdot \Omega
\end{aligned}
\end{equation*}

La constante de fcem est donc : $K_\Phi = 0{,}127$ V/(rad/s).

\vspace{0.5cm}

Une fois le modèle mathématique de la MCC défini à partir des équations présentées ci-dessus, nous pouvons procéder à sa simulation numérique. Cette simulation sera réalisée successivement sur deux environnements logiciels complémentaires : Simulink, qui permet une approche par schéma-blocs, et PSIM, orienté vers la simulation de circuits de puissance. La confrontation des résultats obtenus avec les deux outils permettra de valider notre modélisation théorique.

\subsection{Simulation avec Simulink}

Dans l'environnement Simulink, notre démarche consiste à traduire le système d'équations différentielles sous forme d'un schéma-blocs interconnectés, où chaque bloc représente une opération mathématique élémentaire.

\begin{figure}[H]
    \centering
    \includegraphics[width=1\textwidth]{images/Moteur_a_vide/Schéma moteur à vide Simulink.png}
    \caption{Schéma Simulink du moteur à courant continu à vide}
    \label{fig:schema_moteur_vide_simulink}
\end{figure}
\figref{Figure \ref{fig:schema_moteur_vide_simulink}}{Modélisation\_Intermédiaire\_VALIDE/00\_Moteur\_a\_Vide/Moteur\_Seul\_VALIDE\_12\_09\_2025.slx}

On remarque que le schéma Simulink présenté en Figure \ref{fig:schema_moteur_vide_simulink} intègre les équations de la MCC sans charge. Les simulations effectuées avec ce modèle permettent d'obtenir les réponses temporelles du courant et de la vitesse du moteur à vide, illustrées dans la Figure \ref{fig:reponse_moteur}.

\newpage
\begin{figure}[H]
    \centering
    \begin{minipage}{0.48\textwidth}
        \centering
        \includegraphics[width=\textwidth]{images/Moteur_a_vide/Données courant moteur à vide Simulink.png}
    \end{minipage}
    \hfill
    \begin{minipage}{0.48\textwidth}
        \centering
        \includegraphics[width=\textwidth]{images/Moteur_a_vide/Données vitesse moteur à vide Simulink.png}
    \end{minipage}
    \caption{Résultats de simulation des réponses en courant (gauche) et en vitesse (droite) du moteur à vide}
    \label{fig:reponse_moteur}
\end{figure}
\figref{Figure \ref{fig:reponse_moteur}}{Modélisation\_Intermédiaire\_VALIDE/00\_Moteur\_a\_Vide/Moteur\_Seul\_VALIDE\_12\_09\_2025.slx}

\subsection{Simulation sur PSIM}

Pour la simulation sur PSIM, nous adoptons une approche méthodologique complémentaire. L'utilisation de deux méthodes distinctes (Simulink et PSIM) constitue une validation croisée : si les résultats concordent, nous pourrons confirmer la validité de nos modélisations.
Contrairement à Simulink qui nécessite la construction du modèle à partir des équations différentielles, PSIM propose une approche plus directe basée sur l'utilisation de composants prédéfinis. Le logiciel met à disposition une bibliothèque de machines à courant continu standards qu'il suffit de paramétrer avec les caractéristiques de notre moteur réel. Cette méthodologie se rapproche davantage du fonctionnement physique du système et réduit significativement le nombre de calculs intermédiaires.

\vspace{0.5cm}

Le principe de modélisation sous PSIM repose sur l'assemblage de composants dont on définit ensuite les propriétés. Par exemple, pour simuler notre moteur, nous plaçons simplement un bloc « moteur DC » dans l'environnement de travail, puis nous renseignons ses paramètres électriques et mécaniques.

\begin{figure}[H]
    \centering
    \includegraphics[width=0.7\textwidth]{images/Moteur_a_vide/Schéma moteur à vide Psim.png}
    \caption{Modélisation du moteur à courant continu sur PSIM}
    \label{fig:schema_moteur_psim}
\end{figure}
\figref{Figure \ref{fig:schema_moteur_psim}}{Modélisation\_Intermédiaire\_VALIDE/00\_Moteur\_a\_Vide/Moteur\_Seul\_VALIDE\_12\_09\_2025.psimsch}

\subsection{Comparaison des résultats}
Nous comparons les résultats avec un programme \textit{MATLAB}. Pour cela, nous avons besoin de
récupérer les valeurs de notre modélisation sous forme de tableau : $X = \begin{bmatrix} t & i(t) & \Omega(t) \end{bmatrix}$.

\vspace{0.5cm}
On retrouve bien les valeurs attendues, notamment en vitesse où on a la relation suivante à l'état stable :
\begin{equation*}
\Omega = \frac{U}{K_\Phi} = \frac{48}{0{,}127} = 377{,}95 \text{ rad/s} \quad \text{ou} \quad N = \frac{U}{K_e} = \frac{48}{13,3} = 3609 \text{ tr/min}
\end{equation*}

Les courbes de la Figure \ref{fig:comparaison_resultats_moteur_vide} montrent une excellente concordance entre les résultats obtenus avec Simulink et PSIM, validant ainsi notre modélisation du moteur à courant continu à vide. Nous pouvons donc passer aux simulations suivantes : le moteur avec des pertes et sa charge.

\begin{figure}[H]
    \centering
    \includegraphics[width=0.85\textwidth]{images/Moteur_a_vide/Comparaison courant induit moteur à vide.png}
    
    \vspace{0.5cm}
    
    \includegraphics[width=0.85\textwidth]{images/Moteur_a_vide/Comparaison vitesse moteur à vide.png}
    \caption{Comparaison des résultats Simulink et PSIM pour le moteur à vide}
    \label{fig:comparaison_resultats_moteur_vide}
\end{figure}
\figref{Figure \ref{fig:comparaison_resultats_moteur_vide}}{Modélisation\_Intermédiaire\_VALIDE/00\_Moteur\_a\_Vide/Comparaison\_Moteur\_Seul\_VALIDE\_12\_09\_2025.m}

\newpage
\section{Modélisation de la MCC avec charge}

Dans cette partie, nous allons effectuer la même logique qu'avec le moteur à vide. Cette fois-ci le moteur a des pertes, appelées frottements, et le couple résistant du moteur est pris en compte.

On note :

\begin{equation*}
f = \frac{30}{\pi} \cdot \frac{1}{1\,000} \cdot 0{,}53 \cdot 10^{-2} = 5{,}06 \cdot 10^{-5} \, N.m/rad.s
\end{equation*}

\begin{equation*}
C_r = 2 \cdot C_0 = 0{,}024 \, N.m
\end{equation*}

\subsection{Équations de la MCC avec pertes}

Les équations de notre système vont donc changer. En effet en ajoutant les frottements, le système devient :

\begin{tcolorbox}[colback=remindercolor!5!white, colframe=remindercolor!75!black, title=Formules du moteur avec pertes]
\begin{equation*}
u(t) = R \cdot i(t) + L \frac{di(t)}{dt} + e(t)
\end{equation*}

\begin{equation*}
C_{em}(t) = K_c \cdot i(t)
\end{equation*}

\begin{equation*}
e(t) = K_e \cdot \Omega(t)
\end{equation*}

\begin{equation*}
J \frac{d\Omega}{dt} = C_{em}(t) - 2f \cdot \Omega(t) - C_r(t)
\end{equation*}
\end{tcolorbox}

Ce qui donne dans le domaine de Laplace :

\begin{tcolorbox}[colback=remindercolor!5!white, colframe=remindercolor!75!black, title=Équations dans le domaine de Laplace]
\begin{equation*}
I(s) = \frac{U(s) - K_\Phi \Omega}{R + L \cdot s}
\end{equation*}

\begin{equation*}
C_{em}(s) = K_\Phi \cdot I(s)
\end{equation*}

\begin{equation*}
E(s) = K_\Phi \cdot \Omega(s)
\end{equation*}

\begin{equation*}
\Omega(s) = \frac{C_{em} - C_r}{J \cdot s + f}
\end{equation*}
\end{tcolorbox}

Une fois les équations trouvées, nous allons pouvoir déterminer le schéma bloc et effectuer les simulations.

\newpage
\subsection{Simulation Simulink}

Le schéma bloc n'est plus le même. En effet, celui-ci devient plus complexe, puisque qu'on y ajoute le couple résistant $C_r$ et les frottements $f$.

\begin{figure}[H]
    \centering
    \includegraphics[width=1\textwidth]{images/Moteur_Charge_Frottement/Schéma_Simulink_Moteur_Charge_Frottement.png}
    \caption{Schéma Simulink du moteur à courant continu avec charge et frottements}
    \label{fig:schema_moteur_charge_simulink}
\end{figure}
\figref{Figure \ref{fig:schema_moteur_charge_simulink}}{Modélisation\_Intermédiaire\_VALIDE/01\_Moteur\_Charge\_Frottement/Moteur\_Charge\_Frottement\_VALIDE\_03\_11\_2025.slx}

\subsection{Simulation sur PSIM}

De manière analogue à la simulation du moteur à vide, nous utilisons PSIM pour valider le modèle Simulink incluant cette fois-ci la charge et les frottements. La démarche reste similaire : nous utilisons le bloc moteur à courant continu de PSIM, mais en y ajoutant les paramètres de couple résistant ($C_r$) et de frottement visqueux ($f$). Cette approche directe, basée sur des composants physiques, permet de vérifier la cohérence des résultats obtenus par la modélisation par équations dans Simulink.

\begin{figure}[H]
    \centering
    \includegraphics[width=1\textwidth]{images/Moteur_Charge_Frottement/Schéma_PSIM_Moteur_Charge_Frottement.png}
    \caption{Modélisation du moteur avec charge et frottements sur PSIM}
    \label{fig:schema_moteur_charge_psim}
\end{figure}
\figref{Figure \ref{fig:schema_moteur_charge_psim}}{Modélisation\_Intermédiaire\_VALIDE/01\_Moteur\_Charge\_Frottement/Moteur\_Charge\_Frottement\_VALIDE\_03\_11\_2025.psimsch}

\newpage
\subsection{Comparaison des résultats}

\begin{figure}[H]
    \centering
    \includegraphics[width=0.85\textwidth]{images/Moteur_Charge_Frottement/Graphe Comparaison Courant Moteur.png}
    
    \vspace{0.5cm}

    \includegraphics[width=0.85\textwidth]{images/Moteur_Charge_Frottement/Graphe Comparaison Courant Géné.png}

    \vspace{0.5cm}

    \includegraphics[width=0.85\textwidth]{images/Moteur_Charge_Frottement/Graphe Comparaison Vitesse.png}
    \caption{Comparaison des résultats Simulink et PSIM pour le moteur avec charge}
    \label{fig:comparaison_resultats_moteur_charge}
\end{figure}
\figref{Figure \ref{fig:comparaison_resultats_moteur_charge}}{Modélisation\_Intermédiaire\_VALIDE/01\_Moteur\_Charge\_Frottement/Comparaison\_Courant\_Moteur\_Charge\_Frottement\_VALIDE\_03\_11\_2025.m}

Les courbes de la Figure \ref{fig:comparaison_resultats_moteur_charge} confirment la validité de notre modélisation avec charge et frottements. On observe une superposition quasi-parfaite des résultats Simulink (marqueurs) et PSIM (ligne continue) sur l'ensemble des grandeurs mesurées. La vitesse atteint un régime permanent à environ $3100$ tr/min, valeur inférieure à celle du moteur à vide ($3609$ tr/min) en raison de la présence de la charge de $10$ $\Omega$, du couple résistant $C_r$ et des frottements visqueux. Le courant d'induit du moteur présente un pic initial caractéristique du démarrage, puis se stabilise à une valeur permanente correspondant à l'équilibre entre le couple moteur et les couples résistifs. Cette concordance valide l'implémentation des pertes mécaniques dans les deux environnements de simulation.


\section{Modélisation de la MCC avec charge et hacheur}

\subsection{Modélisation du hacheur}

Le hacheur, ou convertisseur continu--continu, est un dispositif d'électronique de puissance qui permet de modifier la valeur moyenne d'une tension continue à l'aide d'interrupteurs commandés. Il est couramment utilisé pour piloter des charges électriques (ici une machine à courant continu) avec un bon rendement.

Sur PSIM, le hacheur est modélisé par un pont en H composé de quatre transistors MOSFET pilotés en paires. Le schéma fonctionnel comporte le pont en H alimenté par une source continue (ici 48\,V), un générateur de signal triangulaire et un comparateur qui produit le signal PWM de commande.

\begin{figure}[H]
    \centering
    \includegraphics[width=1\textwidth]{images/Moteur_Hacheur/Schéma_PSIM_Moteur_Hacheur.png}
    \caption{Modélisation du hacheur sur PSIM}
    \label{fig:hacheur_psim}
\end{figure}
\figref{Figure \ref{fig:hacheur_psim}}{Modélisation\_Intermédiaire\_VALIDE/02\_Moteur\_Hacheur/Moteur\_Hacheur\_VALIDE\_03\_11\_2025.psimsch}

Le hacheur fournit une tension de sortie moyenne qui peut être positive ou négative (mode bipolaire). La valeur moyenne de la tension de sortie Vs dépend de trois grandeurs : la tension d'alimentation (ici 48\,V), la tension de commande continue (notée $V_{\text{commande}}$) et le rapport cyclique $\alpha$ du signal de commande (duty cycle) appliqué aux interrupteurs.

La commutation des transistors s'effectue par paires (par exemple 1 et 4 d'une part, 2 et 3 d'autre part) de façon complémentaire afin de piloter le pont en H. Le rapport cyclique $\alpha$ correspond à la proportion de la période pendant laquelle l'interrupteur est fermé. Pour $\alpha=0$ la tension de sortie vaut $V_s=-48$\,V, pour $\alpha=1$ on obtient $V_s=+48$\,V, et pour des valeurs intermédiaires de $\alpha$ la tension de sortie varie de manière (approximativement) linéaire entre ces extrêmes. Ainsi, une augmentation de $\alpha$ entraîne une augmentation de $V_s$.

La commande PWM est générée en comparant, au moyen d'un comparateur (ou d'un AOP configuré), une tension continue de référence $V_e = V_{\text{commande}}$ avec un signal triangulaire $V_{\text{triangle}}$ qui varie typiquement de 0 à 15\,V. Le comparateur produit alors le signal $V_{\text{pwm}}$ dont le rapport cyclique est proportionnel à $V_{\text{commande}}$. En sortie du comparateur, des buffers (bootstrappers) sont placés pour assurer une adaptation d'impédance et piloter correctement les étages de commande des MOSFET.

\subsection{Caractéristique du hacheur}

Le hacheur utilisé possède une caractéristique de transfert linéaire. 
La tension de sortie moyenne $\langle u_{AB}\rangle$ dépend de la tension de commande $u_{\text{commande}}$ selon la relation suivante :

\vspace{0.3cm}
\noindent
\begin{minipage}[t]{0.43\textwidth}
\vspace{0pt} % pour aligner le haut
\textbf{Relation tension de sortie :}

\begin{equation*}
\begin{aligned}
\langle u_{AB} \rangle &= \frac{96}{15}\,(u_{\text{commande}} - 7{,}5) \\
&= \frac{96}{15}\,u_{\text{commande}} - 48 \\
&= 6{,}4\,u_{\text{commande}} - 48
\end{aligned}
\end{equation*}

\vspace{0.3cm}
\textbf{Caractéristiques principales :}
\begin{itemize}
    \item Pente : $6{,}4$ V/V
    \item Tension maximale : $48$ V à $15$ V de commande
    \item Point de fonctionnement intermédiaire (en rouge) : $(11{,}25\text{ V} ; 24\text{ V})$
\end{itemize}
\end{minipage}
\hfill
\begin{minipage}[t]{0.55\textwidth}
\vspace{0pt} % même astuce ici
\centering
\begin{tikzpicture}[>=stealth, scale=0.65, xscale=0.5, yscale=0.1]
    % Axes
    \draw[thick,->] (0,-55) -- (0,55) node[above] {$\langle u_{AB}\rangle$ (V)};
    \draw[thick,->] (-1,0) -- (17,0) node[right] {$u_{\text{commande}}$ (V)};
    
    % Graduation sur l'axe y
    \draw[thick] (-0.2,48) -- (0.2,48) node[left=0.3cm] {$48$};
    \draw[thick] (-0.2,-48) -- (0.2,-48) node[left=0.3cm] {$-48$};
    \draw[thick] (-0.2,24) -- (0.2,24) node[left=0.3cm] {$24$};
    \draw[thick] (-0.2,0) -- (0.2,0) node[left=0.3cm] {$0$};
    
    % Graduation sur l'axe x
    \draw[thick] (7.5,-1) -- (7.5,1) node[below=0.5cm] {$7{,}5$};
    \draw[thick] (15,-1) -- (15,1) node[below=0.5cm] {$15$};
    \draw[thick] (11.25,-1) -- (11.25,1) node[below=0.5cm] {$11{,}25$};
    
    % Zone morte
    \draw[thick] (0,0) -- (7.5,0);
    % Droite de montée
    \draw[thick] (7.5,0) -- (15,48);
    % Extension
    \draw[thick] (15,48) -- (16,54.4);
    % Extension de la droite vers la gauche jusqu'à -48V à x=0 
    \draw[thick] (0,-48) -- (7.5,0);
    
    % Point intermédiaire rouge
    \draw[red, dashed, thick] (11.25,0) -- (11.25,24);
    \draw[red, dashed, thick] (0,24) -- (11.25,24);
    \fill[red] (11.25,24) circle (3pt);
    
    % Point max
    \draw[black, dashed, thick] (15,0) -- (15,48);
    \draw[black, dashed, thick] (0,48) -- (15,48);
    \fill[black] (15,48) circle (3pt);
\end{tikzpicture}
\end{minipage}

\vspace{0.3cm}
\begin{figure}[H]
\caption{Caractéristique de transfert du hacheur}
\label{fig:caracteristique_commande}
\end{figure}


\vspace{0.5cm}

Cette caractéristique montre que le hacheur présente un comportement linéaire avec un gain de $6{,}4$. La tension de sortie est limitée à $\pm 48$ V. 

\subsection{Simulation sur Simulink}

\begin{figure}[H]
    \centering
    \includegraphics[width=1\textwidth]{images/Moteur_Hacheur/Schéma_Simulink_Moteur_Hacheur.png}
    \caption{Modélisation du moteur avec hacheur sur Simulink}
    \label{fig:schema_moteur_hacheur_simulink}
\end{figure}
\figref{Figure \ref{fig:schema_moteur_hacheur_simulink}}{Modélisation\_Intermédiaire\_VALIDE/02\_Moteur\_Hacheur/Moteur\_Hacheur\_VALIDE\_03\_11\_2025.slx}

\newpage
\subsection{Comparaison des résultats}

\begin{figure}[H]
    \centering
    \includegraphics[width=0.85\textwidth]{images/Moteur_Hacheur/Graphe Comparaison Courant Géné Hacheur.png}
    \vspace{0.5cm}
    \includegraphics[width=0.85\textwidth]{images/Moteur_Hacheur/Graphe Comparaison Courant Moteur Hacheur.png}
    \vspace{0.5cm}
    \includegraphics[width=0.85\textwidth]{images/Moteur_Hacheur/Graphe Comparaison Vitesse Hacheur.png}
    \caption{Comparaison des résultats Simulink et PSIM pour le moteur avec hacheur}
    \label{fig:comparaison_resultats_moteur_hacheur}
\end{figure}
\figref{Figure \ref{fig:comparaison_resultats_moteur_hacheur}}{Modélisation\_Intermédiaire\_VALIDE/02\_Moteur\_Hacheur/Comparaison\_Courant\_Moteur\_Hacheur\_VALIDE\_03\_11\_2025.m}

Les résultats présentés dans la Figure \ref{fig:comparaison_resultats_moteur_hacheur} démontrent la cohérence entre les modélisations Simulink et PSIM du système complet moteur-hacheur. Les trois grandeurs observées (courant générateur, courant moteur et vitesse) montrent une excellente superposition des courbes issues des deux simulateurs. On constate notamment que l'introduction du hacheur impose une dynamique légèrement différente par rapport à l'alimentation directe, avec des ondulations caractéristiques de la modulation PWM. La vitesse converge vers une valeur de régime permanent proche de $3100$ tr/min, cohérente avec le point de fonctionnement imposé par la tension de commande du hacheur. Cette validation croisée confirme la justesse de notre modélisation du convertisseur statique et de son interaction avec la charge mécanique.