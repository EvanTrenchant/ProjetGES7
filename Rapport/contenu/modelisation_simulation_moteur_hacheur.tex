\chapter{Étude de la MCC et de son hacheur}

\section{Simulation et modélisation du moteur à vide}

\subsection{Équations de la MCC}

La machine à courant continu (MCC) est un dispositif électromécanique capable de convertir l'énergie électrique en énergie mécanique, et inversement. Cette conversion bidirectionnelle s'effectue grâce au passage d'un courant continu dans son induit, générant ainsi un couple moteur qui entraîne l'arbre en rotation. En mode générateur, le principe est inversé : la rotation de l'arbre produit une force électromotrice aux bornes de la machine. La MCC constitue ainsi un élément clé dans de nombreux systèmes d'entraînement industriels et de conversion d'énergie.

\subsubsection{Formules d'une MCC sans pertes et sans charge}

Les équations régissant le comportement d'une machine à courant continu sans pertes et sans charge sont les suivantes :

\begin{tcolorbox}[colback=remindercolor!5!white, colframe=remindercolor!75!black, title=Équations temporelles de la MCC]
\begin{equation*}
u(t) = R \cdot i(t) + L \frac{di(t)}{dt} + e(t)
\end{equation*}

\begin{equation*}
C_m(t) = K_c \cdot i(t)
\end{equation*}

\begin{equation*}
e(t) = K_e \cdot \omega(t)
\end{equation*}

\begin{equation*}
J \frac{d\omega}{dt} = C_m(t)
\end{equation*}

\vspace{0.2cm}
\centering
On note ici : $K_c = K_e = K_\Phi$
\end{tcolorbox}

\subsubsection{Transformée de Laplace}

Les équations précédentes peuvent être exprimées dans le domaine de Laplace, ce qui facilite l'analyse et la modélisation du système :

\begin{tcolorbox}[colback=remindercolor!5!white, colframe=remindercolor!75!black, title=Équations dans le domaine de Laplace]
\begin{equation*}
I(s) = \frac{U(s) - K_\Phi \Omega}{R + Ls}
\end{equation*}

\begin{equation*}
C_m(s) = K_\Phi \cdot I(s)
\end{equation*}

\begin{equation*}
E(s) = K_\Phi \cdot \Omega(s)
\end{equation*}

\begin{equation*}
\Omega(s) = \frac{K_\Phi \cdot I(s)}{Js}
\end{equation*}
\end{tcolorbox}

Une fois le modèle mathématique de la MCC défini à partir des équations présentées ci-dessus, nous pouvons procéder à sa simulation numérique. Cette simulation sera réalisée successivement sur deux environnements logiciels complémentaires : Simulink, qui permet une approche par schéma-blocs, et PSIM, orienté vers la simulation de circuits de puissance. La confrontation des résultats obtenus avec les deux outils permettra de valider notre modélisation théorique.

\subsection{Simulation avec Simulink}

Dans l'environnement Simulink, notre démarche consiste à traduire le système d'équations différentielles sous forme d'un schéma-blocs interconnectés, où chaque bloc représente une opération mathématique élémentaire.

\begin{figure}[H]
    \centering
    \includegraphics[width=0.8\textwidth]{images/moteur_a_vide/Schéma moteur à vide Simulink.png}
    \caption{Schéma Simulink du moteur à courant continu à vide}
    \label{fig:schema_moteur_vide_simulink}
\end{figure}
\figref{Figure \ref{fig:schema_moteur_vide_simulink}}{Modelisation/01\_12\_09\_2025/moteur\_seul\_12\_09\_2025.slx}

On remarque que le schéma Simulink présenté en Figure \ref{fig:schema_moteur_vide_simulink} intègre les équations de la MCC sans charge. Les simulations effectuées avec ce modèle permettent d'obtenir les réponses temporelles du courant et de la vitesse du moteur à vide, illustrées dans la Figure \ref{fig:reponse_moteur}.
\begin{figure}[H]
    \centering
    \begin{minipage}{0.48\textwidth}
        \centering
        \includegraphics[width=\textwidth]{images/moteur_a_vide/Données courant moteur à vide Simulink.png}
    \end{minipage}
    \hfill
    \begin{minipage}{0.48\textwidth}
        \centering
        \includegraphics[width=\textwidth]{images/moteur_a_vide/Données vitesse moteur à vide Simulink.png}
    \end{minipage}
    \caption{Courbes de réponse du moteur à courant continu}
    \label{fig:reponse_moteur}
\end{figure}

\subsection{Simulation sur PSIM}

Pour la simulation sur PSIM, nous adoptons une approche méthodologique complémentaire. L'utilisation de deux méthodes distinctes (Simulink et PSIM) constitue une validation croisée : si les résultats concordent, nous pourrons confirmer la validité de nos modélisations.

Contrairement à Simulink qui nécessite la construction du modèle à partir des équations différentielles, PSIM propose une approche plus directe basée sur l'utilisation de composants prédéfinis. Le logiciel met à disposition une bibliothèque de machines à courant continu standards qu'il suffit de paramétrer avec les caractéristiques de notre moteur réel. Cette méthodologie se rapproche davantage du fonctionnement physique du système et réduit significativement le nombre de calculs intermédiaires.

Le principe de modélisation sous PSIM repose sur l'assemblage de composants dont on définit ensuite les propriétés. Par exemple, pour simuler notre moteur, nous plaçons simplement un bloc « moteur DC » dans l'environnement de travail, puis nous renseignons ses paramètres électriques et mécaniques.

\begin{figure}[H]
    \centering
    \includegraphics[width=0.7\textwidth]{images/moteur_a_vide/Schéma moteur à vide Psim.png}
    \caption{Modélisation du moteur à courant continu sur PSIM}
    \label{fig:schema_moteur_psim}
\end{figure}
\figref{Figure \ref{fig:schema_moteur_psim}}{Modelisation/01\_12\_09\_2025/moteur\_seul\_12\_09\_2025.psimsch}

\subsection{Comparaison des résultats}
Nous comparons les résultats avec un programme \textit{MATLAB}. Pour cela, nous avons besoin de
récupérer les valeurs de notre modélisation sous forme de tableau : $X = \begin{bmatrix} t & i(t) & \omega(t) \end{bmatrix}$.

\vspace{0.5cm}
On retrouve bien les valeurs attendues, notamment en vitesse où on a la relation suivante à l'état stable :
\begin{equation*}
\omega = \frac{U}{K_\Phi} = \frac{48}{0{,}127} = 377{,}95 \text{ rad/s} \quad \text{ou} \quad \omega = \frac{U}{K_e} = \frac{48}{13,3} = 3609 \text{ tr/min}
\end{equation*}

Les courbes de la Figure \ref{fig:comparaison_resultats_moteur_vide} montrent une excellente concordance entre les résultats obtenus avec Simulink et PSIM, validant ainsi notre modélisation du moteur à courant continu à vide. Nous pouvons donc passer aux simulations suivantes : le moteur avec des pertes et sa charge.
\newpage

\begin{figure}[H]
    \centering
    \includegraphics[width=0.85\textwidth]{images/moteur_a_vide/Comparaison courant induit moteur à vide.png}
    
    \vspace{0.5cm}
    
    \includegraphics[width=0.85\textwidth]{images/moteur_a_vide/Comparaison vitesse moteur à vide.png}
    \caption{Comparaison des résultats Simulink et PSIM pour le moteur à vide}
    \label{fig:comparaison_resultats_moteur_vide}
\end{figure}
\figref{Figure \ref{fig:comparaison_resultats_moteur_vide}}{Modelisation/01\_12\_09\_2025/comparasion\_moteur\_seul\_12\_09\_2025.m}

\section{Modélisation du moteur à courant continu avec charge}
\textcolor{red}{cf. Modélisation Simulink et PSIM du dossier 02\_22\_09\_2025}

\section{Modélisation du moteur avec charge et hacheur}
\textcolor{red}{cf. Modélisation Simulink et PSIM du dossier 03\_01\_10\_2025}

\subsection{Conversion vitesse-tension}

Le moteur possède une force contre-électromotrice $E$ proportionnelle à sa vitesse de rotation. 
Les données constructeur indiquent : $E=13{,}3 \mathrm{~V}$ pour $1000 \mathrm{~tr/min}$.

En notant $N$ la vitesse en tr/min et $\Omega$ la vitesse en rad/s, on obtient les relations suivantes :

\begin{equation*}
\begin{aligned}
& E=\frac{13,3}{1000} \cdot N \\
& \Omega=\frac{\pi N}{30} \Rightarrow E=\underbrace{\frac{13,3}{1000} \cdot \frac{30}{\pi}}_{0,127} \cdot \Omega
\end{aligned}
\end{equation*}

La constante de fcem est donc : $K_\Phi = 0{,}127$ V/(rad/s).

\subsection{Caractéristique du hacheur}

Le hacheur utilisé possède une caractéristique de transfert non-linéaire avec une zone morte. 
La tension de sortie moyenne $\langle u_{AB}\rangle$ dépend de la tension de commande $u_{\text{commande}}$ selon la relation suivante :

\vspace{0.3cm}
\noindent
\begin{minipage}[t]{0.43\textwidth}
\vspace{0pt} % pour aligner le haut
\textbf{Relation tension de sortie :}

\begin{equation*}
\begin{aligned}
\langle u_{AB} \rangle &= \frac{96}{15}\,(u_{\text{commande}} - 7{,}5) \\
&= \frac{96}{15}\,u_{\text{commande}} - 48 \\
&= 6{,}4\,u_{\text{commande}} - 48
\end{aligned}
\end{equation*}

\vspace{0.3cm}
\textbf{Caractéristiques principales :}
\begin{itemize}
    \item Zone morte : $0 \leq u_{\text{commande}} < 7{,}5$ V
    \item Pente : $6{,}4$ V/V
    \item Tension maximale : $48$ V à $15$ V de commande
    \item Point de fonctionnement intermédiaire (en rouge) : $(11{,}25\text{ V} ; 24\text{ V})$
\end{itemize}
\end{minipage}
\hfill
\begin{minipage}[t]{0.55\textwidth}
\vspace{0pt} % même astuce ici
\centering
\begin{tikzpicture}[>=stealth, scale=0.65, xscale=0.5, yscale=0.1]
    % Axes
    \draw[thick,->] (0,-55) -- (0,55) node[above] {$\langle u_{AB}\rangle$ (V)};
    \draw[thick,->] (-1,0) -- (17,0) node[right] {$u_{\text{commande}}$ (V)};
    
    % Graduation sur l'axe y
    \draw[thick] (-0.2,48) -- (0.2,48) node[left=0.3cm] {$48$};
    \draw[thick] (-0.2,-48) -- (0.2,-48) node[left=0.3cm] {$-48$};
    \draw[thick] (-0.2,24) -- (0.2,24) node[left=0.3cm] {$24$};
    \draw[thick] (-0.2,0) -- (0.2,0) node[left=0.3cm] {$0$};
    
    % Graduation sur l'axe x
    \draw[thick] (7.5,-1) -- (7.5,1) node[below=0.5cm] {$7{,}5$};
    \draw[thick] (15,-1) -- (15,1) node[below=0.5cm] {$15$};
    \draw[thick] (11.25,-1) -- (11.25,1) node[below=0.5cm] {$11{,}25$};
    
    % Zone morte
    \draw[thick] (0,0) -- (7.5,0);
    % Droite de montée
    \draw[thick] (7.5,0) -- (15,48);
    % Extension
    \draw[thick] (15,48) -- (16,54.4);
    % Extension de la droite vers la gauche jusqu'à -48V à x=0 
    \draw[thick] (0,-48) -- (7.5,0);
    
    % Point intermédiaire rouge
    \draw[red, dashed, thick] (11.25,0) -- (11.25,24);
    \draw[red, dashed, thick] (0,24) -- (11.25,24);
    \fill[red] (11.25,24) circle (3pt);
    
    % Point max
    \draw[black, dashed, thick] (15,0) -- (15,48);
    \draw[black, dashed, thick] (0,48) -- (15,48);
    \fill[black] (15,48) circle (3pt);
\end{tikzpicture}

\captionof{figure}{Caractéristique de transfert du hacheur}
\label{fig:caracteristique_commande}
\end{minipage}


\vspace{0.5cm}

Cette caractéristique montre que le hacheur présente une zone morte pour les faibles tensions de commande (inférieures à $7{,}5$ V), puis un comportement linéaire avec un gain de $6{,}4$. La tension de sortie est limitée à $\pm 48$ V.

